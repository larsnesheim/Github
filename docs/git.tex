\documentclass{beamer}

\usepackage{graphicx}
\usepackage{mathpazo}
\usepackage{hyperref}
\usepackage{multimedia}
\usepackage{epstopdf}
\usepackage{amsmath}
\usepackage{amssymb}
\usepackage{color}
\usepackage{amsfonts}
\usepackage{hyperref}
\usepackage{bbm}

\usepackage{listings}

\definecolor{dkgreen}{rgb}{0,0.6,0}
\definecolor{gray}{rgb}{0.5,0.5,0.5}
\definecolor{mauve}{rgb}{0.58,0,0.82}

\lstset{frame=tb,
  language=Java,
  aboveskip=3mm,
  belowskip=3mm,
  showstringspaces=false,
  columns=flexible,
  basicstyle={\small\ttfamily},
  numbers=none,
  numberstyle=\tiny\color{gray},
  keywordstyle=\color{blue},
  commentstyle=\color{dkgreen},
  stringstyle=\color{mauve},
  breaklines=true,
  breakatwhitespace=true,
  tabsize=3
}

\setbeamertemplate{section page}
{
    \begin{centering}
    \begin{beamercolorbox}[sep=12pt,center]{part title}
    \usebeamerfont{section title}\insertsection\par
    \end{beamercolorbox}
    \end{centering}
}

\usetheme{Madrid}
\usecolortheme{beaver}
% beaver, wolverine, crane

%\usepackage[authoryear,round]{natbib}


\setbeamertemplate{caption}[numbered]
\setbeamertemplate{section page}
{
    \begin{centering}
    \begin{beamercolorbox}[sep=12pt,center]{part title}
    \usebeamerfont{section title}\insertsection\par
    \end{beamercolorbox}
    \end{centering}
}


\title{Beginner tips from an unskilled user for using Git and Github for Version Control}
\author{Lars Nesheim }
\date{November 2023}

\begin{document}

\frame{\titlepage}

\begin{frame}
\frametitle{What is Git?}
\begin{itemize}
\item \textcolor{red}{\textbf{Git}} is a program for simple, efficient \textcolor{red}{\textbf{version control}}:
\begin{itemize}
\item Records changes to files, sets of files.
\item Enables efficient management and control of multiple versions of files.
\end{itemize}
\item Can be used to:
\begin{itemize}
\item Compare \textcolor{red}{\textbf{multiple versions, track changes}} over time, track changes by specific users.
\item Store archive or \textcolor{red}{\textbf{``repository"}} locally, centrally, or \textcolor{red}{\textbf{in cloud}} (e.g. Github).
\item Archive specific versions and/or revert to previous versions.
\item Synchronise versions across \textcolor{red}{\textbf{multiple computers and multiple users.}}
\item Create \textcolor{red}{\textbf{branches}} of projects.
\end{itemize}
\item Works on Linux, Mac, Windows.
\end{itemize}
\end{frame}

\begin{frame}
\frametitle{What is Github?}
\begin{itemize}
\item Github is a \textcolor{red}{\textbf{cloud platform}} owned by Microsoft.
\item Allows for cloud storage of Git repositories.
\begin{itemize}
\item Can be accessed from any computer with web access.
\item Store project repository in cloud.
\item Can easily synchronise with local computers. 
\end{itemize}
\item Allows for either private or public repositories.
\item Many tools for code development, collaboration, project management.
\item Alternatives include Bitbucket and many others.
\end{itemize}
\end{frame}

\begin{frame}
\frametitle{Getting started}
\begin{itemize}
\item Using Git and Github for the first time.
\begin{enumerate}
\item Install Git (if necessary).
\begin{itemize}
\item By default, Git is installed on Linux and Mac (not sure about Windows).
\item On Windows, can download/install Github Desktop.
\end{itemize}
\item Setup Github account.
\begin{itemize}
\item Free personal account has most basic tools/services.
\item UCL account has additional benefits/tools/services.
\end{itemize}
\item Set up secure access to Github.
\begin{itemize}
\item  Use ssh keys.
\item These function like a password to enable password protected access.
\end{itemize}
\end{enumerate}
\end{itemize}
\end{frame}

\begin{frame}
\frametitle{Starting a project}
\begin{enumerate}
\item Create repository on Github
\item Clone repository from Github to local computer.
\item Add, delete, or edit files on local computer.
\item Commit changes (with a brief message describing changes).
\item Push changes from local computer to repository. 
\end{enumerate}
\end{frame}

\begin{frame}
\frametitle{Work on existing project}
\begin{enumerate}
\item Assume local computer already contains copy of remote repository.
\item If necessary, commit any changes to files on local computer.
\item Pull files from remote repository.
\begin{itemize}
\item Option to switch to a different branch.
\end{itemize}
\item If any conflicts, resolve them, commit changes (with message), and push to remote repository.
\item Add, delete, or edit files on local computer.
\begin{itemize}
\item Option to create new branch.
\end{itemize}
\item Commit further changes (with a brief message).
\item Push changes from local computer to repository. 
\end{enumerate}
\end{frame}

\begin{frame}
\frametitle{Getting started}
\begin{itemize}
\item If desired, download
\href{https://desktop.github.com}{\textcolor{blue}{\textbf{Github Desktop}}}.
\item Setup a \href{https://github.com}{\textcolor{blue}{\textbf{Github account}}}.
\item Setup secure access.
\begin{enumerate}
\item Generate \textbf{ssh key} and add to your computer's \textbf{ssh-agent}. 
\href{https://docs.github.com/en/authentication/connecting-to-github-with-ssh/generating-a-new-ssh-key-and-adding-it-to-the-ssh-agent}{\textcolor{blue}{\textbf{Instructions}}}.
\item Add the ssh key to your Github account. 
\href{https://docs.github.com/en/authentication/connecting-to-github-with-ssh/adding-a-new-ssh-key-to-your-github-account}{\textcolor{blue}{\textbf{Instructions}}}.
\end{enumerate}
\end{itemize}
\end{frame}

\begin{frame}[fragile]
\frametitle{Starting a project}
\begin{itemize}
\item Create repository on Github (name, pubilc/private, readme, template, license). Edit readme.
\item Clone repository from Github to local computer.
\begin{lstlisting}
git clone git@github.com:larsnesheim/Github.git
\end{lstlisting}
\begin{itemize}
\item \textbf{command:} \lstinline{git clone} 
\item \textbf{address:} \lstinline{git@github.com:larsnesheim/Github.git} 
\item All files in repository are copied to local computer in new directory ``Github".
\end{itemize}
\end{itemize}
\end{frame}

\begin{frame}
\frametitle{Starting a project: continued}
\begin{table}[htp]
\caption{Other commands}
\begin{center}
\begin{tabular}{ll}
\hline
\lstinline{git status} & Check on current status. \\
\lstinline{git add newfile.txt} & Add new file to project. \\
\lstinline{git rm oldfile.txt} & Remove old file. \\
\lstinline{git commit -a -m "Describe changes made."} & Commit changes. \\
\lstinline{git push} & Copy changes to remote repository.
\end{tabular}
\end{center}
\label{default}
\end{table}%
\begin{itemize}
\item \textbf{Final two steps (1) commit, 2) push) are crucial.}
\end{itemize}
%\begin{tabular}{cc}
%Command & Description \\
%\lstinline{git status} & Check on current status. \\
%\end{tabular}
\end{frame}

\begin{frame}
\frametitle{Work on existing project}
\begin{itemize}
\item \lstinline{git pull} to copy most recent version of repository from remote to local computer.
\item \lstinline{git checkout -b new\_branch} to create a new branch and switch to it.
\item \lstinline{git checkout main} to switch to main branch.
\item \lstinline{git push -u origin new\_branch} to copy new branch to remote repository.
\item \lstinline{git checkout --track origin/branch} to switch to and track ``branch".
\item \lstinline{git merge branch\_to\_merge} to merge current branch and ``branch\_to\_merge".
\end{itemize}
\end{frame}

\begin{frame}
\frametitle{Things to always avoid}
\begin{itemize}
\item Never copy large data files to remote repository.
\item Never copy confidential data to remote repository.
\item Never copy passwords, API tokens, or other confidential information to remote repository.
\end{itemize}
\end{frame}

\begin{frame}
\frametitle{Other topics}
\begin{itemize}
\item Using Github desktop.
\item Using Github.
\item Using submodules.
\item INntegrated with Matlab, RStudio, many IDE's, Overleaf.
\item Finding help:
\begin{itemize}
\item Search on internet.
\item  \href{https://docs.github.com/en/get-started/using-git/about-git}{\textcolor{blue}{\textbf{Github documentation}}}.
\item \href{https://git-scm.com/doc}{\textcolor{blue}{\textbf{Git documentation}}}.
\end{itemize}
\end{itemize}
\end{frame}

\end{document}
